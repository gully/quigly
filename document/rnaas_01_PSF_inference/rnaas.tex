%% rnaastex.cls is the classfile used for Research Notes. It is derived
%% from aastex61.cls with a few tweaks to allow for the unique format required.
\documentclass{rnaastex}

%% Define new commands here
\newcommand\latex{La\TeX}

\newcommand{\vM}{\mathsf{M}}
\newcommand{\vD}{\mathsf{D}}
\newcommand{\vd}{\mathsf{d}}
\newcommand{\vrr}{\mathsf{r}}
\newcommand{\vm}{\mathsf{m}}
\newcommand{\vR}{\mathsf{R}}
\newcommand{\vC}{\mathsf{C}}
\newcommand{\vA}{\mathsf{A}}
\newcommand{\va}{\mathsf{a}}
\newcommand{\vF}{\mathsf{F}}
\newcommand{\vf}{\mathsf{f}}
\newcommand{\vy}{\mathsf{y}}
\newcommand{\trans}{\mathsf{T}}

\begin{document}

\title{Inference of the K2 PSF}

%% Note that the corresponding author command and emails has to come
%% before everything else. Also place all the emails in the \email
%% command instead of using multiple \email calls.
\correspondingauthor{Michael Gully-Santiago}
\email{igully@gmail.com}

\author[0000-0002-4020-3457]{Michael Gully-Santiago}
\affiliation{Kepler/K2 Guest Observer Office \\
NASA Ames Research Center \\
Moffett Blvd \\
Mountain View, CA 94035, USA}
\affiliation{Bay Area Environmental Reaearch Institute}

\author{Gosemite Team and TBD}
\affiliation{TBD}

%% See the online documentation for the full list of available subject
%% keywords and the rules for their use.
\keywords{methods: statistical, techniques: image processing, methods: data analysis}

%% Start the main body of the article. If no sections in the
%% research note leave the \section call blank to make the title.
\section{}

We infer PSFs in Kepler data.  Here is a citation: \cite{2015Natur.521..332O}.

\section{PSF photometry for Kepler}
Astronomical photometry has conventionally followed two paths: aperture photometry or Point Spread Function (PSF) photometry.  Point spread function photometry excels in crowded regions and in low Signal-to-Noise ($S/N$) ratio regimes.  Kepler/K2 mission time-series imaging of star clusters and extended objects will benefit from PSF photometry.  The large computational cost of PSF photometry has hampered its application to crowded fields in Kepler/K2, slowing the unbiased analysis of existing valuable data. In this research note, we lay the foundation for the application of GPU-acceleration applied to PSF photometry, demonstrating a speed-up over conventional computational methods.  We apply our method to both synthetic data, recovering unbiased estimates of our inputs, and real K2 data.  In the latter case we derive the Kepler PSF itself, the individual fluxes of stars, and the covariance among all inferred stars.  Recent advances in GPU acceleration hardware and associated software libraries offer avenues for performance improvements in other domains in astrophysical data analysis.

\section{Methodology}
We make these basic assumptions about our problem structure:

\begin{enumerate}
  \item All sources are point sources.
  \item There are exactly $N_{\star}$ sources.
  \item All sources share the same PSF.
  \item All pixels share the same response function.
  \item The PSF is drawn from a Gaussian Mixture Model with $N_{GMM} = 3$ unknown components.
  \item Each source has 3 parameters-- $x_{\star,i}, y_{\star,i}$ center locations and flux amplitude, $f_i$.
  \item Each PSF component has 5 parameters-- relative $(x_j, y_j)$ location; amplitude $z_j$; and bivariate covariance matrix $C_j$, which is comprised of 3 parameters $(\phi_a, \phi_b, \phi_c)$.
  \item No PSFs are saturated.
  \item All pixels have known, homoscedastic Gaussian read noise $\sigma_{r}$.
  \item The flat field is uniform, or equivalently has already been corrected perfectly.
  \item There are $N_{pix}$ pixels in the scene.
\end{enumerate}

We fix the first Gaussian Mixture Model components $x_1,y_1 \equiv 0, 0$, enforce $z_{j+1} < z_{j}$, and enforce normalization $\int \mathcal{K} \,dx\,dy = 1$. The above assumptions yield a total number of parameters, $N_{\mathrm{param}} = 3 N_\star + 6 N_{GMM} - 3$.

We identified two nearly identical procedures for calculating the likelihood.  In procedure I, we instantiate $N_\star$ delta functions in a sub-sampled pixel grid, convolve a model PSF with the delta functions, and sum the sub-sampled pixels in a Kepler Pixel.  In procedure II, instantiate $N_\star$ model PSFs, evaluating the model at each pixel center.  We anticipate that these two procedures produce the same results, but procedure II is likely to possess smoother derivatives than procedure I, resulting in higher numerical performance.  Procedure II can only be carried out assuming statement number 4 above holds.  We adopted procedure II.  The source of the PSFs can be either source detection in the local image (\emph{e.g. Source Extractor}), or an outside deep catalog in a similar bandpass (\emph{e.g. Gaia}).  We performed source detection for the current note.

The models are:

\begin{eqnarray}
\mathcal{K}_{\mathrm{PSF}} = \sum_{j=0}^{N_{\mathrm{GMM}}} z_j \mathcal{G}(x-x_j, y-y_j, C_j) \\
\hat I(x,y) = \sum_i^{N_\star} f_i \cdot \mathcal{K}_{\mathrm{PSF}}(x-x_{\star, i}, y-y_{\star, i})
\end{eqnarray}

We can rewrite the sums as matrix products.  We assign $ \vd $ as the $N_{pix} \times 1$ column vector mapping all data pixel values from the 2D image onto a 1D vector.  The vector $ \va $ is the 2D PSF model for the $i^{th}$ star mapped onto an $N_{pix} \times 1$ one-dimensional column vector.  The matrix $\vA$ represents the column-wise concatenation of the $N_{\star}$ models $\va_{i}$, yielding an $N_{pix} \times N_{\star}$ matrix.  The vector $\vf$ contains the $N_{\star}$ PSF flux amplitudes $f_i$.  We can analytically compute the ``profile likelihood'' flux amplitudes:

\begin{eqnarray}
\vf^{*} = (\vA^{\trans} \vC^{-1} \vA)^{-1}(\vA^{\trans} \vC^{-1} \vd)
\end{eqnarray}

Computing the $\vf^{*}$ analytically eliminates $N_\star$ linear parameters (the $f_i$'s) from the $N_{\mathrm{param}}$ parameters, and delivers the covariance among the inferred amplitudes.  We derive the non-linear parameters by minimizing $\chi^2$:

\begin{eqnarray}
\vm &=& \vA \cdot \vf^{*} \\
\vrr &=& \vd - \vm \\
  \ln{p(\vd | \vm)} &=& -\frac{1}{2} \left( \vrr^\trans \vC^{-1} \vrr + \ln{\det{\vC}} + N_{pix} \ln{2\pi} \right)
\end{eqnarray}

where $\vC$ is the pixel noise covariance matrix.  For our current assumption of homoscedastic read noise $\vC = \sigma_r \mathcal{I}$ where $\mathcal{I}$ is the diagonal identity matrix.

\section{Application- PSF photometry on synthetic data}

\begin{itemize}
\item Application- PSF photometry
\item Synthetic data generation
\item Application to synthetic data
\item (Stretch goal): Application to real patch of FFI data
\item (Stretch goal): Performance with CPU
\item (Stretch goal): Performance with GPU
\end{itemize}

\section{Appendix}

Here we clarify the notation and offer additional mathematical transformations used in this note.  The normalized Gaussian mixture model takes the form:

\begin{eqnarray}
\mathcal{G} &=& \frac{1}{2 \pi \sqrt{\det{C_j} }} e^{-\frac{1}{2} \vec{r}^{\trans} C_j^{-1} \vec{r}}
\end{eqnarray}

with $\vec{r}=\vec{x}+\vec{y}$ representing the distance vector.  The Kernel and data positions refer to their respective reference frames, with Kernels representing continuous variables and data spaces representing pixel coordinates.  The bivariate Gaussian mixture model covariance matrix can be written down as:
\begin{eqnarray}
   C_j=
  \left[ {\begin{array}{cc}
   e^{2 \phi_a} & \;\;\phi_c e^{\phi_a} \\
   \phi_c e^{\phi_a} & \;\;\phi_c^2 +e^{2\phi_b} \\
  \end{array} } \right]
\end{eqnarray}

which enforces positive scale factors and positive semidefinite matrices.  Enforcing normalization $\int \mathcal{K} \,dx\,dy \equiv 1$ results in $\sum_{j=0}^{N_{\mathrm{GMM}}} z_j = 1$.  We apply several transformations for computational expediency:

\begin{eqnarray}
z_j - z_{j-1} &=& e^{-\theta_j} \\
z_1 &=& \frac{1}{1 + \sum_{j=2}^{N_{\mathrm{GMM}}} z_j} \\
\theta_{j} >0
\end{eqnarray}

%%%%%%%FIGURE%%%%%%%
%\begin{figure}[h!]
%\begin{center}
%        \includegraphics[scale=0.85,angle=0]{figa.pdf}
%\caption{Demo of our K2 PSF inferred from data\label{fig:1}}
%\end{center}
%\end{figure}

\acknowledgments

This research has made use of NASA's Astrophysics Data System.  The reproducible Jupyter Notebook that generated the figures in this document are freely available \href{https://github.com/gully/quigly}{on GitHub}.

\begin{thebibliography}{}

\bibitem[Olling et al.(2015)]{2015Natur.521..332O} Olling, R.~P., Mushotzky, R., Shaya, E.~J., et al.\ 2015, \nat, 521, 332

\end{thebibliography}

\end{document}
