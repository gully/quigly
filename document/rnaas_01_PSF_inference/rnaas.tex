%% rnaastex.cls is the classfile used for Research Notes. It is derived
%% from aastex61.cls with a few tweaks to allow for the unique format required.
\documentclass{rnaastex}

%% Define new commands here
\newcommand\latex{La\TeX}

\begin{document}

\title{Inference of the K2 PSF}

%% Note that the corresponding author command and emails has to come
%% before everything else. Also place all the emails in the \email
%% command instead of using multiple \email calls.
\correspondingauthor{Michael Gully-Santiago}
\email{igully@gmail.com}

\author[0000-0002-4020-3457]{Michael Gully-Santiago}
\affiliation{Kepler/K2 Guest Observer Office \\
NASA Ames Research Center \\
Moffett Blvd \\
Mountain View, CA 94035, USA}
\affiliation{Bay Area Environmental Reaearch Institute}

%% See the online documentation for the full list of available subject
%% keywords and the rules for their use.
\keywords{methods: statistical, techniques: image processing, methods: data analysis}

%% Start the main body of the article. If no sections in the
%% research note leave the \section call blank to make the title.
\section{}

We infer PSFs in Kepler data.  Here is a citation: \cite{2015Natur.521..332O}.

%%%%%%%FIGURE%%%%%%%
%\begin{figure}[h!]
%\begin{center}
%        \includegraphics[scale=0.85,angle=0]{figa.pdf}
%\caption{Demo of our K2 PSF inferred from data\label{fig:1}}
%\end{center}
%\end{figure}

\acknowledgments

This research has made use of NASA's Astrophysics Data System.  The reproducible Jupyter Notebook that generated the figures in this document are freely available \href{https://github.com/gully/quigly}{on GitHub}.

\begin{thebibliography}{}

\bibitem[Olling et al.(2015)]{2015Natur.521..332O} Olling, R.~P., Mushotzky, R., Shaya, E.~J., et al.\ 2015, \nat, 521, 332

\end{thebibliography}

\end{document}
